\documentclass[12pt]{exam}
\usepackage[
  letterpaper,
  margin=0.3in,
  includehead,    % include the header area in the 0.25in margin
  includefoot     % include the footer area in the 0.25in margin
]{geometry}
\usepackage[shortlabels]{enumitem}
\setlist[enumerate,1]{label=(\alph*)} %makes list alphabetic

\usepackage{physics}

\usepackage{pgfplots} % For sketching graphs
\pgfplotsset{compat=1.18}


\usepackage{caption}

\title{Introduction to Quantum Mechanics - Solution Manual \newline \small{For Griffiths and Schroeter's Third Edition}}

\author{Justin Beltran}
\date{May 2025}

\begin{document}

\maketitle
\firstpagefooter{}{}{By Justin Beltran}
\runningfooter{}{}{By Justin Beltran}
\printanswers

\section{Chapter 1}



\subsection*{Problem 1.1}
For the distribution of ages in the example in Section 1.3.1
\begin{enumerate}[(a)]
    \item  Compute $\expval{j^2}$ and $\expval{j}^2$

    \item Determine $\Delta j$ for each $j$, and use Equation 1.11 to compute the standard deviation.

    \item Use your results in (a) and (b) to check Equation 1.12
\end{enumerate}

\begin{solution}
    \begin{enumerate}[(a)]
        \item There is one person aged 14, one person aged 15, three people aged 16, two people aged 22, two people aged 24, five people aged 25 then 
        $$\expval{j^2} = \sum_j j^2 P(j) = 14^2(\frac{1}{14}) + 15^2(\frac{1}{14}) + 16^2(\frac{3}{14})+ 22^2(\frac{2}{14})+ 24^2(\frac{2}{14})+ 25^2(\frac{5}{14})$$
        $$= \frac{196}{14} + \frac{225}{14}+ \frac{768}{14}+ \frac{968}{14}+ \frac{1152}{14}+ \frac{3125}{14} = \frac{6434}{14} = 459.57$$
        and 
        $$\expval{j}^2 =  \big[\sum_j j^ P(j) = 14(\frac{1}{14}) + 15(\frac{1}{14}) + 16(\frac{3}{14})+ 22(\frac{2}{14})+ 24(\frac{2}{14})+ 25(\frac{5}{14})\big]^2$$
        $$= \big[ \frac{14}{14}+ \frac{15}{14} + \frac{48}{14} + \frac{44}{14} + \frac{48}{14} + \frac{125}{14}\big]^2 = \big[\frac{294}{14}\big]^2 = 21^2 = 441$$

        \item $\expval{j}=21$ then $\Delta j_{14} = 14-21 = -7$, $\Delta j_{15} = 15-21 = -6$, $\Delta j_{16} = 16-21 = -5$,  $\Delta j_{22} = 22-21 = 1$, $\Delta j_{24} = 24-21 = 3$, $\Delta j_{25} = 25-21 = 4$.  Plugging into 1.11 $$\sigma^2 = \expval{(\Delta j)^2}  = (-7)^2(\frac{1}{14}) + (-6)^2(\frac{1}{14}) + (-5)^2(\frac{3}{14})+ 1^2(\frac{2}{14})+ 3^2(\frac{2}{14})+ 4^2(\frac{5}{14})$$ 
        $$
        = 49(\frac{1}{14}) + 36(\frac{1}{14}) + 25(\frac{3}{14})+ (\frac{2}{14})+ 9(\frac{2}{14})+ 16(\frac{5}{14})= \frac{259}{14} = 18.57 \implies\sigma = 4.31$$

        \item Equation 1.12 states $\sigma = \sqrt{\expval{j^2}-\expval{j}^2}$ so plugging in answers from part (a) yields $\sigma =\sqrt{459.57-441} = 18.57$ which is as expected.
    \end{enumerate}
\end{solution}






\end{document}
